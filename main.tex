\documentclass[a4paper, 11pt]{article}
\usepackage[utf8]{inputenc}
\usepackage{microtype}
\usepackage{graphicx}
\graphicspath{ {./images/} }
\usepackage[font=small, justification=centering, labelfont=bf]{caption}
\usepackage{subcaption}
\usepackage[a4paper, top=2cm, bottom=2cm, left=2cm, right=2cm, footnotesep=0.8cm]{geometry}
\usepackage{csquotes}
\MakeOuterQuote{"}
\usepackage[hyphens]{url}
\usepackage{hyperref}
\usepackage{cite}

\begin{document}

\title{Advanced Interaction Design}
\author{birthen }
\date{February 2020}



\maketitle

\section{Introduction}

%Personas:
%Write 4-5 different personas 
Personas:
A rich picture of an imaginary person who represents your core user group
Based on actual studies of users, observation, interviews etc. 

“User models, or personas, are fictional, detailed archetypal characters that represent distinct groupings of behaviours, goals and motivations observed and identified during research phase”

Personas:
May have several personas to represent key stakeholder users
Used to focus design on user needs 
Stops getting fixated on ‘general’ user




Persona 1: - Child\\
5 y/o 
Bertie Nansen 
Currently learning about the alphabet in school
Likes dogs and video games


Persona 2: - Colour blind person (Protanopia)\\
45 y/o
Anton Guson
Scared of flying
Likes to watch action movies

Persona 3: - Old person\\
76 y/o
Jenny Duram
Some cognitive disabilities related to age
Finds watching videos to be tiring (prefers audio books, magazines)

Persona 4: - Regular Person\\
37 y/o
Hillary Coucon\\
Big fan of snacks
Likes to stay updated on the news 

%Scenarios: 
1. Buy snacks / drinks / goodies\\
2. Watch movies / TV Shows\\
3. See current plane location\\
4. Play games\\
5. Reading news or books\\


Scenario = story of a typical use
Problems Scenarios: used to identify the problem domain
Activity Scenarios: explore the user and their task more directly
Information Scenarios: help transform the activity scenarios into scenarios that help design the HCI
Interaction Scenarios: describe the actual interaction with the interface




\end{document}

